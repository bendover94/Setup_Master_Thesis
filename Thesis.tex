%%%%%%%%%%%%%%%%%%%%%%%%%%%%%%%%%%%%%%%% Klasse Festlegen
\documentclass[Master,MMR,english]{BASE/twbook} 
%\documentclass[Bachelor,BMR,english,fhCitStyle,IEEE]{BASE/twbook} % FH definierte Zitierstandards verwenden 
%%%%%%%%%%%%%%%%%%%%%%%%%%%%%%%%%%%%%%%% Verwendete Packages
\usepackage[utf8]{inputenc} % Zeichen-Enkodierung (evtl. Abweichungen für Apple)
\usepackage[T1]{fontenc}    % Zeichen-Enkodierung
\usepackage{blindtext}      % Platzhaltertexte
\usepackage{minted}         % Darstellung von Code
\usepackage{comment}        % Auskommentieren von ganzen Passagen
\usepackage{csquotes}
\usepackage{algorithm}      % Umgebung f Algorithmen
\usepackage[noend]{algpseudocode}
                            % Wenn Sie während Ihrer Arbeit
                            % merken, dass Sie zusätzliche Funktionen
                            % benötigen ist hier ein guter Platz um
                            % weitere Packages zu laden
%%%%%%%%%%%%%%%%%%%%%%%%%%%%%%%%%%%%%%%% Zitierstil zum selbst definieren
\usepackage[backend=biber, style=ieee]{biblatex}            % LaTeX definierter IEEE- Standard
%\usepackage[backend=biber, style=authoryear]{biblatex}      % LaTeX definierter Harvard-Standard
\addbibresource{Literatur.bib}                              % Literatur-File definieren
%%%%%%%%%%%%%%%%%%%%%%%%%%%%%%%%%%%%%%%% Einträge für Deckblatt
\title{Arbeitstitel\\Arbeitstitel}

\author{Titel Vorname Name, Titel}
\studentnumber{XXXXXXXXXXXXXXX}
%\author{Titel Vorname Name, Titel\and{}Titel Vorname Name, Titel}
%\studentnumber{XXXXXXXXXXXXXXX\and{}XXXXXXXXXXXXXXX}

\supervisor{Titel Vorname Name, Titel}
%\supervisor[Begutachter]{Titel Vorname Name, Titel}
%\supervisor[Begutachterin]{Titel Vorname Name, Titel}
%\secondsupervisor{Titel Vorname Name, Titel}
%\secondsupervisor[Begutachter]{Titel Vorname Name, Titel}
%\secondsupervisor[Begutachterinnen]{Titel Vorname Name, Titel}

\place{Wien}
%%%%%%%%%%%%%%%%%%%%%%%%%%%%%%%%%%%%%%%% Danksagung/Kurzfassung/Schlagworte
\kurzfassung{\blindtext}
\schlagworte{Schlagwort1, Schlagwort2, Schlagwort3, Schlagwort4}
\outline{\blindtext}
\keywords{Keyword1, Keyword2, Keyword3, Keyword4}
\acknowledgements{\blindtext}
\setListingsAndAcronyms % Definition der Namen für Quellcodeverzeichnis 
%%%%%%%%%%%%%%%%%%%%%%%%%%%%%%%%%%%%%%%% Ende des Headers
%%%%%%%%%%%%%%%%%%%%%%%%%%%%%%%%%%%%%%%% Beginn des Dokuments
\begin{document}
%%%%%%%%%%%%%%%%%%%%%%%%%%%%%%%%%%%%%%%% 
\maketitle
%%%%%%%%%%%%%%%%%%%%%%%%%%%%%%%%%%%%%%%% Beginn des Inhalts
\chapter{Erste Überschrift der Tiefe 1 (chapter)}
Etwas Text... Hier kommen noch einige Abkürzunge vor zum Beispiel \ac{ABC},\ac{WWW} und \ac{ROFL}.

\section{Erste Überschrift Tiefe 2 (section)}
\blindtext

\subsection{Erste Überschrift Tiefe 3 (subsection)}
\blindtext

\subsubsection{Erste Überschrift Tiefe 4 (subsubsection)}
\blindtext

\chapter{Zweite Überschrift der Tiefe 1 (chapter)}
\blindtext

\section{Zweite Überschrift Tiefe 2 (section)}
\blindtext

\subsection{Zweite Überschrift Tiefe 3 (subsection)}
\blindtext

\subsection{Dritte Überschrift Tiefe 3 (subsection)}
\blindtext

\subsubsection{Zweite Überschrift Tiefe 4 (subsubsection)}
\blindtext

\noindent Querverweise werden in \LaTeX{} automatisch erzeugt und verwaltet, damit sie leicht aktualisiert werden können. Hier wird zum Beispiel auf Abbildung \ref{Abb1} verwiesen.

\begin{figure}[!htbp]
\centering
\includegraphics[width=0.5\linewidth]{PICs/buchruecken}
\caption{Beispiel für die Beschriftung eines Buchrückens.}\label{Abb1}
\end{figure}
\begin{figure}[!htbp]
\centering
\includegraphics[width=0.5\linewidth]{PICs/buchruecken}
\caption{2. Beispiel für die Beschriftung eines Buchrückens.}\label{Abb2}
\end{figure}

Und hier ist ein Verweis auf Tabelle \ref{tab1}. Das gezeigte Tabellenformat ist nur ein Beispiel. Tabellen können individuell gestaltet werden.

\begin{table}[!htbp]
\centering
\caption{Semesterplan der Lehrveranstaltung \glqq Angewandte Mathematik\grqq.}\label{tab1}
\begin{tabular}{| p{0.3\linewidth} | p{0.3\linewidth} | p{0.3\linewidth} |}\hline
Datum & Thema & Raum\\\hline
20.08.2008 & Graphentheorie & HS 3.13\\
01.10.2008 & Biomathematik & HS 1.05\\\hline
\end{tabular}
\end{table}
\begin{table}[!htbp]
\centering
\caption{2. Semesterplan der Lehrveranstaltung \glqq Angewandte Mathematik\grqq.}\label{tab2}
\begin{tabular}{| p{0.3\linewidth} | p{0.3\linewidth} | p{0.3\linewidth} |}\hline
Datum & Thema & Raum\\\hline
20.08.2008 & Graphentheorie & HS 3.13\\
01.10.2008 & Biomathematik & HS 1.05\\\hline
\end{tabular}
\end{table}

Hier wird auf die Formel \ref*{Gl1} verwiesen.

\begin{align}
x = -\frac{p}{2}\pm\sqrt{\frac{p^2}{4}-q}\label{Gl1}
\end{align}
\begin{align}
x = -\frac{p}{2}\pm\sqrt{\frac{p^2}{4}-q}\label{Gl2}
\end{align}

Literaturverweise sollten automatisch verwaltet werden, vor allem, wenn es viele Quellenverweise gibt. Beispiele sind  \cite{Ko05a}, \cite{Ko05b}, \cite{MiGo05}, \cite{TeGo14}, \cite{HuHa07}, \cite{HuZi10}, \cite{ZiKu07}, \cite{He07}, \cite{SIE11}, \cite{SIE14}, \cite{ISO98}, \cite{ATM11}, \cite{Hu11}, \cite{Po10}. Das verwendete Zitierformat (bzw.~das Format des Literaturverzeichnisses) ist entspechend der Vorgaben der Studiengänge zu wählen.
%%%%%%%%%%%%%%%%%%%%%%%%%%%%%%%%%%%%%%%%%%%%%%%%%%%%%%%%%%%%%%%%%%
\chapter{Dritte Überschrift der Tiefe 1 (chapter)}
Hier wird etwas Quellcode dargestellt:
\begin{listing}[htbp]
\begin{minted}[
    frame=single,
    framesep=2mm,
    baselinestretch=1.2,
    bgcolor=white,
    fontsize=\footnotesize,
    linenos
    ]{c}
#include <iostream>

void SayHello(void)
{
    // Kommentar
    cout << "Hello World!" << endl;
}

int main(int argc, char **argv)
{
    SayHello();
    return 0;
}
\end{minted}
\caption{Hello-World}
\end{listing}


\section{Algorithms}


Use a defined environment for algorithms.

Algorithm \ref{alg:euclid} is an example from the gallery (\url{https://www.overleaf.com/latex/examples/euclids-algorithm-an-example-of-how-to-write-algorithms-in-latex/mbysznrmktqf}) .
%%%%%%%%%%%%%%%%%%%%%%%%%%%%%%%%%%%%%%%%%%%%%%%%%%%%%%%%%%%%%%%%%%
\begin{algorithm}
\caption{Euclid’s algorithm}\label{alg:euclid}
\begin{algorithmic}[1]
\Procedure{Euclid}{$a,b$}\Comment{The g.c.d. of a and b}
\State $r\gets a\bmod b$
\While{$r\not=0$}\Comment{We have the answer if r is 0}
\State $a\gets b$
\State $b\gets r$
\State $r\gets a\bmod b$
\EndWhile\label{euclidendwhile}
\State \textbf{return} $b$\Comment{The gcd is b}
\EndProcedure
\end{algorithmic}
\end{algorithm}
%%%%%%%%%%%%%%%%%%%%%%%%%%%%%%%%%%%%%%%%%%%%%%%%%%%%%%%%%%%%%%%%%% Hier beginnen die Verzeichnisse.
\clearpage                                                       % Beginne neue Seite

\printbib                                                        % Literaturverzeichnis LaTeX-Zitier-Standard
%\printbib{Literatur}                                             % Literaturverzeichnis FH-Zitier-Standard
\clearpage

\listoffigures                                                   % Abbildungsverzeichnis
\clearpage

\listoftables                                                    % Tabellenverzeichnis
\clearpage

\listoflistings                                                  % Quellcodeverzeichnis
\clearpage

\phantomsection
\addcontentsline{toc}{chapter}{\listacroname}
\chapter*{\listacroname}
\begin{acronym}[XXXXX]
    \acro{ABC}[ABC]{Alphabet}
    \acro{WWW}[WWW]{world wide web}
    \acro{ROFL}[ROFL]{Rolling on floor laughing}
\end{acronym}
%%%%%%%%%%%%%%%%%%%%%%%%%%%%%%%%%%%%%%%%%%%%%%%%%%%%%%%%%%%%%%%%%% Hier beginnt der Anhang.
\clearpage
\appendix
\chapter{Anhang A}
\clearpage
\chapter{Anhang B}
\end{document}
%%%%%%%%%%%%%%%%%%%%%%%%%%%%%%%%%%%%%%%%%%%%%%%%%%%%%%%%%%%%%%%%%% Ende des Inhalts